\chapter{接口缺陷静态检测技术研究}
\label{cha:imchecker}
对接口缺陷检测进行研究具有重要意义。
一方面,接口误用是导致软件错误、系统崩溃、漏洞产生的重要愿意之一。
另一方面,随着开源社区的发展,软件库文档缺失、开发人员对API理解不足,
导致现有的代码中存在大量的接口误用缺陷。
0.5页

\section{引言}
静态分析是什么(cxP65)
1页

现有某一类

规模化

\section{相关工作}

静态分析(cxP67)
具体工具表格介绍
2页

总结的表格
1页
给出表格(tsinghua)

不足0.5页


\section{接口缺陷静态检测算法}
(seke19)

总体图和流程

每部分的主要工作
1.5页

\subsection{构造分析上下文}
编译抓取,为什么IR

介绍CFA,给一个图
1页

多入口分析策略cx89
0.5页

函数展开
循环
0.5

\subsection{抽象符号路径提取}
语法

解释用例子

结果
1.5页
\subsection{缺陷检测算法}
算法
1页
例子解释
0.5页
\begin{algorithm}
	\KwIn{~~~\textsf{SyncBlock} 计算模型内所有复合构件的 $<connection>$,
		\\~~~~~~~~~~~~~~~顶层复合构件的所有输入端口 $<port\_declaration>$,
		\\~~~~~~~~~~~~~~~顶层的每一个输入端口 $p_i$ 对应的环境输入值 $I_i$。 }
	\KwOut{更新所有与顶层复合构件输入端口相连接的端口的值。}
	\vspace{3ex}
	%\SetAlgoLined
	\For{~~顶层复合构件 $<port\_declaration>$ 中的每一个输入端口 $p_i$~~}{
		$p_i \gets I_i$ \;
		\tcp{取出所有与 $p_i$ 直接相连的内部子构件的端口}
		$p_i \left[\right] \gets Connection\_Target(p_i,~<connection>)$ \;
		\For{~~$p_i \left[\right]$ 中的每个端口 $p_i\left[ j \right]$~~} {
			\tcp{根据端口 $p_i\left[ j \right]$ 的性质,判断是否要继续向下搜索传递}
			\eIf{~~$p_i\left[ j \right]$ 不是原子构件的输入端口~~}{
				\tcp{如果连接上有表达式,进行数据处理再赋值}
				\eIf{~~$<connection>$~上有表达式~~}{
					$p_i\left[ j \right] \gets Comput\_Exp(p_i)$ \;
				}{
					$p_i\left[ j \right] \gets p_i$ \;
				}
				\tcp{取出所有与$p_i\left[ j \right]$ 直接相连的端口,添加到 $p_i \left[\right]$ 中}
				$p_{temp} \left[\right] \gets Connection\_Target(p_i\left[ j \right], <connection>)$ \;
				$p_i \left[\right] \gets Append(p_i \left[\right], p_{temp} \left[\right])$ \;
				
			}{
				\tcp{$p_i\left[ j \right]$ 是原子构件的输入端口,不需要继续向下搜索传递}
				\eIf{~~$<connection>$~上有表达式~~}{
					$p_i\left[ j \right] \gets Comput\_Exp(p_i)$ \;
				}{
					$p_i\left[ j \right] \gets p_i$ \;
				}
			}
			
		}
	}
	\caption{从外部环境读入输入序列,传递到原子构件输入端口}
	\label{alg:Input}
\end{algorithm}

\subsection{检测结果过滤}
为什么

语义例子

统计例子
1.5页
\section{工具实现与实验评估}
(tase19)
\subsection{工具实现}
0.5
\subsection{实验环境介绍}
测试集合

每类特点介绍
1页
比较对象
0.5
\subsection{评测结果}
详细分析

结果表格解释1页

自己工具每个分类的情况2页,误报、漏报原因

其他工具对比2页


\section{本章小结}
