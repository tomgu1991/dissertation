\begin{resume}

  \resumeitem{个人简历}

  1991 年 06 月 02 日出生于 辽宁 省 大连 市。

  2010 年 9 月考入 南开 大学 软件学院  软件工程 专业,2014 年 7 月本科毕业并获得 工学 学士学位。

  2014 年 9 月免试进入 清华 大学 软件 学院攻读 博士 学位至今。

  \researchitem{发表的学术论文} % 发表的和录用的合在一起

  % 1. 已经刊载的学术论文(本人是第一作者,或者导师为第一作者本人是第二作者)
  \begin{publications}
    %\item Yang Y, Ren T L, Zhang L T, et al. Miniature microphone with silicon-
    %  based ferroelectric thin films. Integrated Ferroelectrics, 2003,
    %  52:229-235. (SCI 收录, 检索号:758FZ.)
    \item \textbf{Gu Z}, Jiang Y, et al. A Cyber-Physical System Framework for Early Detection of Paroxysmal Diseases[J]. IEEE Access, 2018, 6: 34834-34845. (SCI 收录, 检索号:GN6RX,影响因子4.199)
    \item Wang Y, \textbf{Gu Z}, et al. A constraint-pattern based method for reachability determination[C]//Proceedings of the 41st IEEE Annual Computer Software and Application Conference, Turin, Italy, 2017: 85-90. (CCF-C 类会议,EI 检索号:20174304306998)
    \item \textbf{Gu Z}, Song H, et al. An integrated Medical CPS for early detection of paroxysmal sympathetic hyperactivity[C]//2016 IEEE International Conference on Bioinformatics and Biomedicine (BIBM), Shenzhen, China, 2016, pp. 818-822. (CCF-B 类会议,EI 检索号:20170803377769)
  \end{publications}

  % 2. 尚未刊载,但已经接到正式录用函的学术论文(本人为第一作者,或者
  %    导师为第一作者本人是第二作者)。
  \begin{publications}[before=\publicationskip,after=\publicationskip]
    %\item Yang Y, Ren T L, Zhu Y P, et al. PMUTs for handwriting recognition. In
    %  press. (已被 Integrated Ferroelectrics 录用. SCI 源刊.)
    \item \textbf{Gu Z}, Wu J C, et al. Vetting API Usages in C Programs with IMChecker. In press. (已被41th International Conference on Software Engineering录用,CCF-A 类会议)
    \item \textbf{Gu Z}, Zhou M, et al. IMSpec: An Extensible Approach to Exploring the Incorrect Usage of APIs. In press. (已被13th International Symposium on Theoretical Aspects of Software Engineering录用,CCF-C 类会议)
    \item \textbf{Gu Z}, Wu J C, et al. An Empirical Study on API-Misuse Bugs in Open-Source C Programs. In press. (已被 43rd IEEE Annual Computer Software and Application Conference录用,CCF-C 类会议)
    \item \textbf{Gu Z}, Wu J C, et al. SSLDoc: Automatically Diagnosing Incorrect SSL API Usages in C Programs. In press. (已被 The 31st International Conference on Software Engineering \& Knowledge Engineering录用,CCF-C 类会议)
  \end{publications}


  \researchitem{参与的科研项目}
  \begin{achievements}
  	\item 2016.4-2016.12:国家自然科学基金重大集成项目“可信嵌入式软件系统试验环境与示范应用”(No.91218302)
  	\item 2016.9 至今:科技部国家重点研发计划“规模化漏洞分析与应用关键技术研究”(No.2016QY07X1402)
  	
  	
  \end{achievements}

  \researchitem{研究成果} % 有就写,没有就删除
  \begin{achievements}
    %\item 任天令, 杨轶, 朱一平, 等. 硅基铁电微声学传感器畴极化区域控制和电极连接的
    %  方法: 中国, CN1602118A. (中国专利公开号)
    \item TsmartV3工具集:http://tsmart.tech
    \item APIMU4C数据集:https://github.com/imchecker/compsac19
  \end{achievements}

\end{resume}
