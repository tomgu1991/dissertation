\begin{figure}[t]
	\centering
\begin{lstlisting}
"hash": "d285b5418ee1ff361f06545e0489ece61bdd1a50",
(*@\textcolor{black}{"summary": "Avoid a double-free in crl2pl7",}@*)
"date": "2016-06-14 11:27:10",
"author": "Matt Caswell",
apps/crl2p7.c
=======================================================
lhs: 100644 | 1631258793ec96ae5d653be3b6ebb7bfa7f3c26d
rhs: 100644 | 9c5f79f9f37988d0db709eda6eeaf82d44a1044a
@@ -84,10 +84,8 @@ int crl2pkcs7_main(int argc, char **argv)


if ((certflst == NULL) && (certflst = sk_OPENSSL_STRING_new_null()) == NULL)
	goto end;
-if (!sk_OPENSSL_STRING_push(certflst, opt_arg())) {
-	sk_OPENSSL_STRING_free(certflst);
+if (!sk_OPENSSL_STRING_push(certflst, opt_arg()))
	goto end;
-}

end:
	sk_OPENSSL_STRING_free(certflst);
\end{lstlisting}
	\caption{
	OpenSSL中重复释放内存对象导致的ICC-API误用缺陷(sha:d285b5418e)
	}
	\label{fig:2-3-icc-3}
\end{figure}