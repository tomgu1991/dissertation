\begin{figure}[t]
	\centering
	\begin{lstlisting}
// IMSpec for fopen, which opens the file specified in the first parameter. If fails, a NULL pointer will be return.
(*@\textcolor{blue}{Spec:}@*)
(*@\textcolor{blue}{Target:}@*) fopen(char*, char*) -> FILE*
(*@\textcolor{blue}{Pre:}@*) 
- fopen_arg_1 != NULL, 
- fopen_arg_2 IN (r, w, a, r+, w+, a+)
(*@\textcolor{blue}{Post:}@*) 
// failure status and error handling actions specific in example
- fopen_arg_0 == NULL, RETURN(FILEERR);
// success status, close file handler
- fopen_arg_0 != NULL, CALL(fclose: fopen_arg_0 == fclose_arg_1)

// IMSpec for fgets, which reads charactors and stores them. If a read error occurs, a null pointer is returned.
(*@\textcolor{blue}{Spec:}@*)
(*@\textcolor{blue}{Target:}@*) fgets(char*, int, FILE*) -> char*
(*@\textcolor{blue}{Pre:}@*) // omit single parameter validation
- LEN(fgets_arg_1) >= fgets_arg_2 
(*@\textcolor{blue}{Post:}@*)
// failure status 
- fgets_arg_0 == NULL, CALL(log: true), RETURN(IOERR)
	\end{lstlisting}
	\caption{
		FreeRDP中忽略调用关系中参数和返回值的语义关系导致的ICC-API误用缺陷(sha:1845c0b590)。
	}
	\label{fig:2-4-example-imspec}
\end{figure}