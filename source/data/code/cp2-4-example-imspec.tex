\begin{figure}[t]
	\centering
	\begin{lstlisting}
// IMSpec for fopen, which opens the file specified in the first parameter. If fails, a NULL pointer will be return.
(*@\textcolor{blue}{Spec:}@*)
(*@\textcolor{blue}{Target:}@*) fopen(char*, char*) -> FILE*
(*@\textcolor{blue}{Pre:}@*) 
(*@\textcolor{black}{- fopen\_arg\_1 != NULL, }@*)
(*@\textcolor{black}{- fopen\_arg\_2 IN (r, w, a, r+, w+, a+)}@*)
(*@\textcolor{blue}{Post:}@*) 
// failure status and error handling actions specific in example
(*@\textcolor{black}{- fopen\_arg\_0 == NULL, RETURN(foo:FILEERR);}@*)
// success status, close file handler
(*@\textcolor{black}{- fopen\_arg\_0 != NULL, CALL(fclose: fopen\_arg\_0 == fclose\_arg\_1)}@*)

// IMSpec for fgets, which reads charactors and stores them. If a read error occurs, a null pointer is returned.
(*@\textcolor{blue}{Spec:}@*)
(*@\textcolor{blue}{Target:}@*) fgets(char*, int, FILE*) -> char*
(*@\textcolor{blue}{Pre:}@*) // omit single parameter validation
(*@\textcolor{black}{- LEN(fgets\_arg\_1) >= fgets\_arg\_2 }@*)
(*@\textcolor{blue}{Post:}@*)
// failure status 
(*@\textcolor{black}{- fgets\_arg\_0 == NULL, CALL(log: true), RETURN(\_:IOERR)}@*)
	\end{lstlisting}
	\caption{
		针对于图~\ref{fig:2-4-example}中目标接口的IMSpec实例。
	}
	\label{fig:2-4-example-imspec}
\end{figure}