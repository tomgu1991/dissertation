\chapter{绪论}
\label{cha:intro}

\section{研究背景}
在过去的二十年里,软件在我们的日常生活中无处不在。
每一天,新开发的软件服务为我们带来独特的生活视野,使我们的生活更加便捷与高效。
同时,软件开发为经济发展带来了巨大贡献。
据非盈利国际软件研究组织Software.org~\cite{software-org}评估,软件行业为美国2016年国内生产总值贡献超过1.14万亿美元,提供就业岗位超过1050万人。与过去两年相比,平均增长6.5\%~\cite{2017-eco-report}。
在我国,2017年软件和信息技术服务业发展迅速,收入到达5.5万亿元,同比增长14.2\%,对全国发展指数增长贡献率为58.7\%~\cite{2018-china-report}。


如此庞大的软件产业建立在不断增长的软件产品基础之上。因此需要软件开发公司持续的快速研发,以减少每种新产品的上市时间。
例如通过采用持续部署技术,Facebook公司每天发布的新产品数量能够达到100,甚至1000次~\cite{16-icse-continuous}。
与此同时,随着软件需求的增加,当今的软件系统越来越复杂。
为能够匹配市场需求的开发速度,至关重要的一点就是对软件组件的重用~\cite{2011-icsr-reuse, 2013-cbse-reuse},即软件库。
这些库为现代软件提供了构建块产品,重用它们可以让开发人员站在巨人的肩膀上,专注于创新,而不是重塑基本模块。
更具体地说,这些软件库通常提供应用程序编程接口(API)以供开发者来快速构建软件系统。


软件开发者在使用这些API时,需要掌握这些接口的语义和约束条件以正确使用这些编程接口,完成目标任务。
一个API使用(API usage)是一段使用某个或某些API来完成特定功能的代码片段。
通常是一些基本程序元素(program element)的组或。
例如:函数调用,条件判断,算数运算等等。
这些程序元素的组合需要满足目标API的使用规约(usage constraint),
例如:对目标API参数的检查、API调用顺序关系和异常处理等等。
这些使用规约则和API自身属性相关。
当一个API使用违反了规约中的一个或多个时,我们称为不正确的API使用,即API误用(API misuse)~\cite{16-msr-mubench}。
API误用,已成为导致软件缺陷、崩溃,甚至漏洞的重要原因之一~\cite{12-ccs-android,12-ccs-ssl,13-ccs-misuse,13-tse-missing-call,14-apsys-case,15-icpc-api,16-ase-spec}。


例如图~\ref{fig:1-1-example}展示了漏洞CVE-2015-0288~\cite{CVE-2015-0288}的代码片段。
OpenSSL~\cite{openssl}是广泛应用的安全通信软件库。
该库将加密套接字协议层(SSL)~\cite{ssl}中的通信协议和加密算法封装在API中,以方便客户端开发者使用。
开发者在使用OpenSSL提供的API时,需要对各种证书(Certificate)进行有效性验证,以确保信息的安全性和有效性。
比如,函数\texttt{X509\_to\_X509\_REQ()}用于解码证书,当发生错误是,其返回NULL作为错误代码。
因此当开发者忽略该检查时,则会导致一个空指针引用错误(图中第9行所示)。
攻击者可以通过构造非法的证书,利用该漏洞进行拒绝服务攻击(Denial of Service),造成目标系统崩溃。



\definecolor{dkgreen}{rgb}{0,0.6,0}
\definecolor{gray}{rgb}{0.5,0.5,0.5}
\definecolor{blue}{rgb}{0,0,1}
\definecolor{green}{rgb}{0,0.8,0}
\definecolor{mauve}{rgb}{0.58,0,0.82}
\definecolor{light-gray}{gray}{0.25}
\definecolor{numberColor}{rgb}{0.5,0.5,0.5}
\definecolor{backgroundColour}{rgb}{0.95,0.95,0.92}
\definecolor{diffstart}{rgb}{0.5,0.5,0.5}
\definecolor{diffincl}{rgb}{0,1,0}
\definecolor{diffrem}{rgb}{1,0,0}
\definecolor{black}{rgb}{0,0,0}

\lstset{
	language=C,
	float=[tb],
	aboveskip=0.5em,
	belowskip=0em,
	showstringspaces=false,
	columns=flexible,
	basicstyle={\scriptsize\ttfamily\bfseries},
	numberstyle=\color{numberColor},
	numbers=left,
	xleftmargin=20.0ex,
	% frame=b,
	keywordstyle=\color{blue},
	commentstyle=\color{gray},
	breaklines=true,
	breakatwhitespace=true,
	tabsize=4, 
	sensitive = true,
	morestring=*[d]{"},
	morestring=[s][]{\#\{}{\}},
	morecomment=[f][\color{diffstart}]{@},
	morecomment=[f][\color{diffstart}]{file},
	morecomment=[f][\color{diffstart}]{index},
	morecomment=[f][\color{diffincl}]{+},
	morecomment=[f][\color{diffrem}]{-},
	morecomment=[f][\color{black}]{Log},
	escapeinside={(*@}{@*)},
	float=ht,
	% captionpos=b,
}

\newfontfamily\listingsfont[Scale=.7]{DejaVu Sans Mono}
\begin{figure}
	
	\centering
\begin{lstlisting}
	========== Incorrect Usage ==========
Location: OpenSSL/crypto/x509/x509_req.c: 70
X509_REQ *X509_to_X509_REQ(...){
	[(*@\dots@*)]
	(*@\textcolor{mauve}{pktmp = X509\_get\_pubkey(x);}@*)
	// missing certificate validation of pktmp
	(*@\textcolor{green}{+ if (pktmp == NULL)}@*)
	(*@\textcolor{green}{+ ~~~~goto err;}@*)
	i = X509_REQ_set_pubkey(ret, (*@\textcolor{mauve}{pktmp}@*));
	EVP_PKEY_free(pktmp);
	[(*@\dots@*)]
}

	========== Correct Usage ==========

Location: /crypto/x509/x509_cmp.c: 390
int X509_chain_check_suiteb(...){
	[(*@\dots@*)]
	(*@\textcolor{mauve}{pk = X509\_get\_pubkey(x);}@*)
	rv = check_suite_b((*@\textcolor{mauve}{pk}@*), -1, &tflags);
	[(*@\dots@*)]
} 

static int check_suite_b(EVP_PKEY *(*@\textcolor{mauve}{pkey}@*),...){
	[(*@\dots@*)]
	// ensure pkey not NULL
	(*@\textcolor{mauve}{if (pkey \&\& ...)}@*)
	[...]// error handling
}
\end{lstlisting}
	\caption{
	缺失参数检查导致的漏洞CVE-2015-0288~\cite{CVE-2015-0288}
	}
	\label{fig:1-1-example}
\end{figure}


为帮助使用者理解库函数的作用和使用约束,设计人员通过各种方式提供提供高质量的文档以及应用案例共使用者参考。
然而这些辅助材料在现有的开发环境下,并没有有效的帮助开发者理解这些约束~\cite{09-icse-doc}。
更严重的是,随着开源软件的蓬勃发展,大量的库函数没有完整的文档资料,甚至没有或者存在错误的使用说明~\cite{15-ieee-doc-fail, 17-icse-api-doc}。
此外,当遇到API使用问题时,开发者更喜欢直接在搜索引擎或者问答论坛Stack-Overflow~\cite{stackoverflow}中进行查询,而不是查看官方使用说明。
不幸的是,这些问答论坛中存在大量的错误~\cite{16-sp-stack}。
研究表明,如何正确使用API是阻碍开发者开发的重要瓶颈之一~\cite{16-icse-cry}。


因此,研究人员采用更加主动的方式协助开发者正确使用API,即软件工程推荐系统(RSSE)~\cite{10-ieee-rsse}。
该方法的核心是,在软件工程任务上下文中,自动获得重要的信息并提供给开发者,以提高开发者的开发效率。
一方面,通过相似代码检索~\cite{05-icse-rec,16-icse-doc-stack,14-msr-stack}提供API使用样例,或者通过自动补全技术辅助使用者进行开发~\cite{15-tosem-code-cplt}。
另一方面,研究人员通过自动化的分析方法对已有代码进行缺陷检测,推荐潜在的API误用缺陷,以提高代码质量。


在过去的十几年中,大量的研究成果被成功应用于自动化API缺陷检测~\cite{15-coufless-static-survey,18-icse-saful,survey18}。
特别地,静态分析技术只需要源代码,可以在开发的早期进行而获得了广泛关注~\cite{05-icse-static}。
尽管在API误用检测方面,研究人员投入了大量的工作,API误用在实际项目中依旧普遍存在~\cite{16-ase-spec, 18-icse-stack}。
一项对Google应用商店的调研显示,在11748个安卓应用中,10327个超过88\%的被测对象存在至少一个加密相关的API误用~\cite{13-ccs-misuse}。
特别地,富有经验的开发者的代码中~\cite{18-soups-api-blind},甚至对缺陷修复的代码中,同样会出现API误用缺陷。
例如,OpenSSL的开发在创建了缺陷补丁(sha: 1c4221)以``修复crl2pkcs7 app中存在的内存泄漏缺陷''。
然而该补丁忽略上下文路径关系,导致在某些可达路径上对``内存重复释放''。
该缺陷被新的补丁(sha: d285b5)修复,修复的日志为``在crl2pl7中避免重复释放缺陷''。


随着软件库函数数量增加、软件规模增长与代码复杂度提高、开源代码广泛应用,现有的API误用缺陷检测方法面临巨大挑战。
一方面,缺陷检测需要具有较强检测能力,即尽可能多地检测出代码库中的实际缺陷。
另一方面,缺陷检测需要具有较高精度。如果一个检测工具具有大量误报,即缺陷报告中的缺陷并不是实际缺陷,开发者将会摒弃这些工具~\cite{10-acm-precision}。
同时,对于检测结果,为开发者提供有效的缺陷展示形式,提高开发者对缺陷修复效率也是促进检测工具应用的重要因素之一~\cite{13-icse-donotuse}。
因此,如何设计有效的API缺陷静态检测技术,以应对大规模复杂代码,为开发人员提供高精度、低漏报的缺陷检测服务,并能够辅助开发者理解缺陷提高修复效率,是目前工业界迫切需求,也是学术界的研究热点。







%本章将介绍C程序接口缺陷的概念,总结C程序接口缺陷的关键问题与研究现状,说明研究思路与主要贡献。

%API广泛(18P17)

%什么是API误用(18P27)

%API检测难点(18P25)

\section{研究现状}
(review)
代码审查

动态分析

静态分析

验证技术

现有技术不足

\section{研究思路}
整体思路图(cxP13)

本文的研究思路具体如下:

1.

2.

3.



\section{论文贡献}
本文根据上述研究思路,针对C程序接口缺陷的静态检测技术进行了系统性研究,取得了相应的理论效果,实现了对应的工具软件集合。论文的具体贡献如下:

1. 为XXX,做了XXX。

2.

3.
\section{论文结构}
本文共包括5个章节。第2章介绍针对接口使用的规约描述语言,包括基于实际案例的接口缺陷分类以及规约描述语言的设计思路、语言和语义。
第3章讨论接口缺陷的静态检测技术。
第4章介绍接口缺陷检测工具集合与开源项目的应用。
最后第5章总结全文并展望延伸本文工作的若干方向。
