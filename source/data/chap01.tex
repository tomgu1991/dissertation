\chapter{绪论}
\label{cha:intro}

\section{研究背景}
(tase19)
A~\cite{IEEE-1363}

API广泛(18P17)

什么是API误用(18P25)

实例

API检测的重要性

\section{研究现状}
(review)
代码审查

动态分析

静态分析

验证技术

现有技术不足

\section{研究思路}
整体思路图(cxP13)

本文的研究思路具体如下:

1.

2.

3.



\section{论文贡献}
本文根据上述研究思路,针对C程序接口缺陷的静态检测技术进行了系统性研究,取得了相应的理论效果,实现了对应的工具软件集合。论文的具体贡献如下:

1. 为XXX,做了XXX。

2.

3.
\section{论文结构}
本文共包括5个章节。第2章介绍针对接口使用的规约描述语言,包括基于实际案例的接口缺陷分类以及规约描述语言的设计思路、语言和语义。
第3章讨论接口缺陷的静态检测技术。
第4章介绍接口缺陷检测工具集合与开源项目的应用。
最后第5章总结全文并展望延伸本文工作的若干方向。
