\chapter{接口缺陷检测工具集与应用}
\label{cha:tools}
0.25页
(jyP92)

\section{引言}
0.25页
(jyP92)

\section{C程序接口误用数据集}

\paragraph{Gitgrabber}
0.5页
是什么

实现原理

目的

\paragraph{接口缺陷实例}
0.5页
是什么

实现原理

目的

\paragraph{接口缺陷测试集}
0.5页
是什么

实现原理

目的

\section{工具集组成}
(icse-tool)
\subsection{总体架构}
1页
\subsection{程序预处理模块}
1页
\subsection{规约撰写模块}
1页
\subsection{缺陷检测模块}
1页
\subsection{结果展示模块}
1页
差异性结果展示模块

1. 规约 2. 对比。源于两方面,sec2.5被测对象认为IMSpec更有效。
实际开发者学习到的经验,即对路径和对比的需求。

\section{案例应用}
\label{sec:4.4}
\subsection{应用对象}
1页
(icse-tool)
对象

目标API

https://github.com/tomgu1991/IMChecker/tree/master/imspec

平台参数



\subsection{缺陷检测结果}
%https://github.com/tomgu1991/IMChecker/blob/master/evaluation_data/new_bugs/bug_list.md

总体结果
0.5
buglist

\paragraph{Linux内核}
2页

\paragraph{OpenSSL}
2页

\paragraph{应用程序}
2页

每类缺陷举例子
代码+回复


\subsection{开源社区反馈}
%https://github.com/tomgu1991/IMChecker/blob/master/evaluation_data/new_bugs/bug_list.md
1. 感谢
0.5页

2. 新的Bugchecker
0.5页

3. keepalive的后续
0.5页

\subsection{应用经验总结}

3页

\section{本章小结}
1页