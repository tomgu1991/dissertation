\chapter{接口使用规约描述语言研究}
\label{cha:impsec}
随着软件规模和复杂度的提升与开源社区的蓬勃发展,
开发者经常利用现有的应用编程接口(API)来快速构建系统。
在使用API时完成特定功能时,
需要满足对应的约束条件,
例如:检查参数的有效性,正确的调用序列等等。
违反这些约束中的一条或多条,
则会产生接口误用(API misuse),
导致程序缺陷、系统崩溃,甚至被攻击者利用。
如何对接口使用规约进行描述,
是避免接口误用的前提条件。
接口行为规约描述语言(BISL)提供了面向代码层次的形式化描述,
能够有效的帮助程序员理解API的行为以及使用条件。
因此,针对接口误用的缺陷特点,
通过领域特定规约描述语言能够有效的定义接口使用约束。


本章首先针对不同领域的开源C程序中的接口缺陷实例进行分析,
总结接口缺陷常见模式。
接着基于缺陷模式,提出轻量级接口使用规约描述语言IMSpec,
描述了该语言的设计动机,定义了该语言的语法结构和语义信息。
本章将IMSpec应用于调研中收集的接口缺陷实例,以验证语言的有效性。
从全文的研究体系上看,本章的工作旨在通过形式化的方法对接口使用约束条件描述
是接口缺陷检测工作的重要基础。


\section{引言}
开发者在使用API构造软件系统时,
需要满足特定的使用约束条件以正确的完成相应的功能。
例如:API参数为指针类型是,该指针不可以为空,
否则产生一个空指针错误;
当通过内存申请API获得内存资源后,需要使用相应的释放API以规约资源,
否则产生一个内存泄漏错误。
这些由于误用API产生的缺陷是软件缺陷、系统崩溃的重要原因之一,
甚至会被攻击者利用,带来巨大影响。


为了保证API的正确使用,
一方面,API的设计者提供了各种各样的文档、应用案例,
以帮助使用者理解API的功能和对应的使用约束条件,
然而,现在的文档形式难以保证API被正确使用~\cite{09-icse-doc}。
更严重的是,随着开源软件的蓬勃发展,
大量的库函数没有完整的文档资料,
甚至没有或者存在错误的使用说明~\cite{15-ieee-doc-fail, 17-icse-api-doc}。
相对于直接查找官方的API使用文档,
更多的使用者通过网络搜索来快速的找到相应的使用方法。
另一方面,研究人员通过缺陷检测的方法对API误用进行查找,
以提高代码质量。
然而现有的检测方法难以满足实际需求。
(1)基于静态分析技术的检测工具,
多通过预先实现的检测器来进行缺陷查找~\cite{15-coufless-static-survey}。
因此,该方法难以找到为定义的API缺陷。
(2)基于数据挖掘技术的方法,通过推理API使用规约,
再基于规约来进行缺陷检测。
然而,现有的数据集质量难以满足学习算法的数据要求~\cite{survey18}。
无论是API的设计人员还是缺陷检测的研究人员,
如何有效的定义API使用规约是保证接口正确使用的基础。

BISL提供了面向代码层次的形式化描述,
能够有效的帮助程序员理解API的行为以及使用条件~\cite{survey12}。
通俗来说,这些规约描述为接口的开发者和使用者提供了一种形式化的契约模式(software contract)~\cite{92-ieee-contract}。
这些规约描述通过形式化的方法,
指令为能够获得正确的结果,
API在使用时需要满足的特定的约定(convention)。
然而,针对于API使用约束的描述,
现有的BISL具有若干不足。
(1)现有的普适性程序特征的BISL多基于接口的实现而设计,
即有利于描述API的内部属性。
随着软件的规模和复杂性增加,API的使用情景复杂化。
普适性的BISL难以方便的描述API使用的约束条件。
(2)现有的针对接口使用的BISL往往只关注某个特定的领域,
语义的表达能力不足,难以应用到普适性的API使用。
例如:SLIC~\cite{01-slic}针对于Windows驱动程序设计,SSLINT~\cite{15-sp-sslint}针对于SSL的若干API设计。


本章XXX。

本章其余部分组织结构如下:首先在

\section{相关工作}
通用语言

特殊应用语言

API-misuse bug study(18P34)

compsac19

\section{接口缺陷分类}
(compsac19)
\subsection{数据收集}
\subsection{分类结果}

\section{规约描述语言}
\subsection{设计动机}
\subsection{语法}
\subsection{语义}

\section{应用案例}

\section{本章小结}
本章对