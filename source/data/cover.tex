\thusetup{
  %******************************
  % 注意:
  %   1. 配置里面不要出现空行
  %   2. 不需要的配置信息可以删除
  %******************************
  %
  %=====
  % 秘级
  %=====
  secretlevel={秘密},
  secretyear={10},
  %
  %=========
  % 中文信息
  %=========
  ctitle={C程序接口缺陷\\静态检测技术研究},
  cdegree={工学博士},
  cdepartment={软件学院},
  cmajor={软件工程},
  cauthor={谷祖兴},
  csupervisor={顾明教授},
  % cassosupervisor={孙家广教授}, % 副指导老师
  %ccosupervisor={某某某教授}, % 联合指导老师
  % 日期自动使用当前时间,若需指定按如下方式修改:
  % cdate={超新星纪元},
  %
  % 博士后专有部分
  cfirstdiscipline={计算机科学与技术},
  cseconddiscipline={系统结构},
  postdoctordate={2009年7月——2011年7月},
  id={编号}, % 可以留空: id={},
  udc={UDC}, % 可以留空
  catalognumber={分类号}, % 可以留空
  %
  %=========
  % 英文信息
  %=========
  etitle={Static Analysis Based API-Misuse Bug Detection in C Programs},
  % 这块比较复杂,需要分情况讨论:
  % 1. 学术型硕士
  %    edegree:必须为Master of Arts或Master of Science(注意大小写)
  %             “哲学、文学、历史学、法学、教育学、艺术学门类,公共管理学科
  %              填写Master of Arts,其它填写Master of Science”
  %    emajor:“获得一级学科授权的学科填写一级学科名称,其它填写二级学科名称”
  % 2. 专业型硕士
  %    edegree:“填写专业学位英文名称全称”
  %    emajor:“工程硕士填写工程领域,其它专业学位不填写此项”
  % 3. 学术型博士
  %    edegree:Doctor of Philosophy(注意大小写)
  %    emajor:“获得一级学科授权的学科填写一级学科名称,其它填写二级学科名称”
  % 4. 专业型博士
  %    edegree:“填写专业学位英文名称全称”
  %    emajor:不填写此项
  edegree={Doctor of Philosophy},
  emajor={Software Engineering},
  eauthor={Gu Zuxing},
  esupervisor={Professor Gu Ming},
  % eassosupervisor={Professor Sun Jiaguang},
  % 日期自动生成,若需指定按如下方式修改:
  % edate={December, 2005}
  %
  % 关键词用“英文逗号”分割
  %ckeywords={API误用;静态分析;调研;缺陷检测},
  %ekeywords={API misuse; Static analysis; Empirical study; Bug detection}
}

% 定义中英文摘要和关键字
\begin{cabstract}
  TODO:\\
  问题\\
  目的\\
  主要内容包括:
  \begin{itemize}
  	\item 1;
  	\item 2;
  	\item 3。
  \end{itemize}
  
  本文的研究成果均已集成于可信软件工具集TsmartV3中,并成功应用于开源软件。在Linux内核、OpenSSL加密库、Ubuntu应用软件等项目中,发现TODO个实际缺陷,其中TODO个已经被开发者修复。
\end{cabstract}

% 如果习惯关键字跟在摘要文字后面,可以用直接命令来设置,如下:
\ckeywords{API误用;静态分析;调研;缺陷检测}

\begin{eabstract}
   TODO
\end{eabstract}

\ekeywords{API misuse; Static analysis; Empirical study; Bug detection}
